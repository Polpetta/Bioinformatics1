\section{Alignment of amino acid sequences}

Sometimes, during evolution, in a protein some amino acids may be replaced 
by other amino acids, usually with similar properties. Some other times one or 
more amino acids may be added or removed. 

Therefore when aligning protein (and also DNA) sequences we should consider 
the possibility of amino acid substitution, but also insertions and deletions.
These alignment gaps are generally represented by a " - ".

How can we align sequences?
We require algorithms able to find the best alignment between two sequences,
allowing gaps and mismatches.
Problems:
\begin{itemize}
  \item We need a(N) objective function to establish exactly what we want to obtain
  \item We need an algorithm to find the best alignment, that is the alignment
that returns the highest score with the objective function. Alternatively, an
algorithm to find an alignment that approximate in the best possible way the
best alignment.
\end{itemize}

\subsection{Matches and mismatches have different weights}

We require algorithms able to find the best alignment between two sequences,
allowing gaps and mismatches.

Mismatches are not all the same. Some amino acids are very similar and can 
replace each other with little consequences; other amino acids are very different 
and should have different weight.

A logical consequence of the above is that also matches should have different 
weight on the alignment.

We need a computational evaluation of the alignment. First we need an objective 
function to establish in mathematical terms what we want to obtain. 
Then we need a suitable algorithm to find the alignment that maximizes the 
value of the objective function.

The following \textbf{objective function} is generally used to align biological 
sequences: 

\begin{equation}
Score = \sum_{i=1}^{L} s(a_i,b_i) - \sum_{j=1}^{G} (\gamma + \delta(len(j)-1))
\end{equation}

\begin{itemize}
	\item \textbf{L} is the length of the alignment
	\item The \textbf{sum $ s(a_i,b_i) (i=1..L) $} is calculated from the \textbf{substitution 
matrix} s that defines a score for every pair of aligned amino acids.
	\item Then comes the \textbf{penalty for gaps}. Each penalty is calculated for each 
individual gap. \textbf{$\gamma$} is the fixed penalty for the opening of the gap and then 
there is a penalty proportional to to gap length \textbf{$\delta$}.
\end{itemize}
