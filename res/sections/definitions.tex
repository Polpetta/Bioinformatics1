\newpage
\section{Useful definitions}

\begin{itemize}
	\item \textbf{Diploid cells} contain two complete sets (2n) of
chromosomes
	\item \textbf{Haploid cells} have half the number of chromosomes
(n) as diploid/they contain only one complete set of chromosomes
	\item \textbf{mRNA} (messenger RNA) contains the informations to encode
proteins for the translation process
	\item An \textbf{Allele} is the variant form of a given gene
	\item \textbf{Homozygous} means having identical pairs of genes for any 
given pair of hereditary characteristics.
	\item In diploid organisms, \textbf{heterozygous} refers to having two
different alleles for a specific gene.
	\item \textbf{Meiosis} is a specialized type of cell division that
reduces the chromosome number by half, creating four haploid cells, each
genetically distinct from the parent cell. Meiosis occurs in all sexually
reproducing single-celled and multicellular eukaryotes
	\item \textbf{Introns and exons} are parts of genes.
Exons code for proteins, whereas introns do not. Exons are converted into
mRNA
	\item \textbf{Transcription} is the process by which DNA is used as a
template to create mRNA
\end{itemize}

\subsection{Answer to theory questions}

\paragraph*{How is called the enzyme that duplicate the DNA?} DNA Polymerase
\paragraph*{How is called the enzyme that transcribes DNA into RNA?} RNA Polymerase
\paragraph*{How is it called the biological structure where proteins are
synthesized?} Ribosome
