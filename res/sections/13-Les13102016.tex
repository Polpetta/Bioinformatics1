\section{Approaches}
\subsection{Reductionistic approach}
This approach say that a organism is far too complex to understand everything
in one time. The redustionistic approach try to understand everything by
"splitting" the apparatus in small parts. Since there are so many different
component we need to annotate this component using ontologies and databases.

The second part of this approach is to put this information togheter to make
sense of the living organism.

\section{Genom sequences}
The instruments for DNA sequencing produce relatively short fragments of
sequence (reads), 100 to 1000 bases long, depending on the technology.
For example:
\begin{itemize}
  \item 1 human genom $\sim3$ Gbp
  \item 1 run $\to$ $\sim120$ Gbp reads $\approx150$ bases each
\end{itemize}

The DNA have 4 bases.
Protein have 20 amminoacids.

Some example of DNA:
\begin{verbatim}
  5' AATCCG 3'
  3' TTAGGC 5'
       =
  5' CGGATT 3'
  3' GCCTAA 5'
\end{verbatim}
